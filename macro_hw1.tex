\documentclass{article} % For LaTeX2e
\usepackage{nips14submit_e,times}
\usepackage{amsmath}
\usepackage{amsthm}
\usepackage{amssymb}
\usepackage{mathtools}
\usepackage{hyperref}
\usepackage{url}
\usepackage{algorithm}
\usepackage[noend]{algpseudocode}
%\documentstyle[nips14submit_09,times,art10]{article} % For LaTeX 2.09

\usepackage{graphicx}
\usepackage{caption}
\usepackage{subcaption}

\def\eQb#1\eQe{\begin{eqnarray*}#1\end{eqnarray*}}

\providecommand{\e}[1]{\ensuremath{\times 10^{#1}}}
\providecommand{\pb}[0]{\pagebreak}


\newenvironment{claim}[1]{\par\noindent\underline{Claim:}\space#1}{}
\newtheoremstyle{quest}{\topsep}{\topsep}{}{}{\bfseries}{}{ }{\thmname{#1}\thmnote{ #3}.}
\theoremstyle{quest}
\newtheorem*{definition}{Definition}
\newtheorem*{theorem}{Theorem}
\newtheorem*{question}{Question}
\newtheorem*{exercise}{Exercise}
\newtheorem*{challengeproblem}{Challenge Problem}
\newtheorem*{solution}{Solution}
\usepackage{verbatimbox}
\usepackage{listings}
\title{Intro to Macroeconmics: Assignment I}


\author{
Youngduck Choi, Noah Gentile, Yuliya Takh \\
New York University \\
}


% The \author macro works with any number of authors. There are two commands
% used to separate the names and addresses of multiple authors: \And and \AND.
%
% Using \And between authors leaves it to \LaTeX{} to determine where to break
% the lines. Using \AND forces a linebreak at that point. So, if \LaTeX{}
% puts 3 of 4 authors names on the first line, and the last on the second
% line, try using \AND instead of \And before the third author name.

\newcommand{\fix}{\marginpar{FIX}}
\newcommand{\new}{\marginpar{NEW}}

\nipsfinalcopy % Uncomment for camera-ready version

\begin{document}


\maketitle

\begin{abstract}
\end{abstract}

\section{Solutions to the problems}

\begin{question}[1. Review Question (2)]
\end{question}
\begin{solution}
The concept of opportunity cost is 
\end{solution}

\bigskip

\begin{question}[3. Utility function]
\end{question}
\begin{solution}
\textbf{(1)} We claim that preferences are strictly monotonic with respect to $z$ and $s$.
Taking the partials with respect to $z$ and $s$, we obtain
\eQb
\dfrac{\partial U}{\partial z} &=& \dfrac{\alpha}{2} z^{-\frac{1}{2}}, \\
\dfrac{\partial U}{\partial s} &=& \dfrac{1 - \alpha}{2} s^{-\frac{1}{2}}. \\
\eQe
As $\alpha \in (0,1)$, we see that both partials are strictly positive for all positive
values of $z$ and $s$, which is the domain of interest in this case. Hence, we have shown
that the preferences are strictly monotonic with respect to $z$ and $s$. Note 
that strict monotonicity implies monotonicity as well.

\smallskip

\textbf{(2)} 

\smallskip

\textbf{(3)} 
Yes, the marginal utility is decreasing. Evaluating the second-order partials
from the first-order partials obtained in part $(1)$, we get
\eQb
\dfrac{\partial^2 U}{\partial z^2} &=& -\dfrac{\alpha}{4} z^{-\frac{3}{2}}, \\
\dfrac{\partial^2 U}{\partial s^2} &=& -\dfrac{1 - \alpha}{4} s^{-\frac{3}{2}}. \\
\eQe
As $\alpha \in (0,1)$, we see that for any positive values of $z$ and $s$, we have 
\eQb
\dfrac{\partial^2 U}{\partial z^2} & < & 0, \\
\dfrac{\partial^2 U}{\partial s^2} & < & 0. \\
\eQe
Therefore, we have shown that the marginal utility is decreasing with 
respect to both variables. It is, in fact,
strictly decreasing.

\smallskip

\textbf{(4)}

\smallskip

\textbf{(5)}

\smallskip

\textbf{(6)}

\smallskip

\textbf{(7)}

\smallskip

\textbf{(8)}


\end{solution}


\end{document}
