\documentclass{article} % For LaTeX2e
\usepackage{nips14submit_e,times}
\usepackage{amsmath}
\usepackage{amsthm}
\usepackage{amssymb}
\usepackage{mathtools}
\usepackage{hyperref}
\usepackage{url}
\usepackage{algorithm}
\usepackage[noend]{algpseudocode}
%\documentstyle[nips14submit_09,times,art10]{article} % For LaTeX 2.09

\usepackage{graphicx}
\usepackage{caption}
\usepackage{subcaption}

\def\eQb#1\eQe{\begin{eqnarray*}#1\end{eqnarray*}}

\providecommand{\e}[1]{\ensuremath{\times 10^{#1}}}
\providecommand{\pb}[0]{\pagebreak}


\newenvironment{claim}[1]{\par\noindent\underline{Claim:}\space#1}{}
\newtheoremstyle{quest}{\topsep}{\topsep}{}{}{\bfseries}{}{ }{\thmname{#1}\thmnote{ #3}.}
\theoremstyle{quest}
\newtheorem*{definition}{Definition}
\newtheorem*{theorem}{Theorem}
\newtheorem*{question}{Question}
\newtheorem*{exercise}{Exercise}
\newtheorem*{challengeproblem}{Challenge Problem}
\newtheorem*{solution}{Solution}
\usepackage{verbatimbox}
\usepackage{listings}
\title{Intro to Macroeconmics: Problem Set IV}


\author{
Group 8: Youngduck Choi, Noah Gentile, Yuliya Takh \\
New York University \\
}


% The \author macro works with any number of authors. There are two commands
% used to separate the names and addresses of multiple authors: \And and \AND.
%
% Using \And between authors leaves it to \LaTeX{} to determine where to break
% the lines. Using \AND forces a linebreak at that point. So, if \LaTeX{}
% puts 3 of 4 authors names on the first line, and the last on the second
% line, try using \AND instead of \And before the third author name.

\newcommand{\fix}{\marginpar{FIX}}
\newcommand{\new}{\marginpar{NEW}}

\nipsfinalcopy % Uncomment for camera-ready version

\begin{document}


\maketitle

\begin{abstract}
This document contains the solutions to the problem set IV.
\end{abstract}

\section{Solutions to the problems}

\begin{question}[1]
\end{question}
\begin{solution}
\textbf{(a)}
We are given the demand function as follow:
\eQb
C_1^{A} &=& \dfrac{1 + \rho_A}{2 + \rho_A} (Y_1^{A} + \dfrac{Y_2^{A}}{1 + r }), \\
C_1^{B} &=& \dfrac{1 + \rho_B}{2 + \rho_B} (Y_1^{B} + \dfrac{Y_2^{B}}{1 + r }), \\
C_2^{A} &=& \dfrac{1 + r}{2 + \rho_A} (Y_1^{A} + \dfrac{Y_2^{A}}{1 + r }), \\
C_2^{B} &=& \dfrac{1 + r}{2 + \rho_B} (Y_1^{B} + \dfrac{Y_2^{B}}{1 + r }). \\
\eQe
We have that $S = Y_1 - C_1$. Therefore, we can compute $S$s as
\eQb
S_A &=& Y_1^{A} - C_1^{A}, \\
S_B &=& Y_1^{B} - C_1^{B},
\eQe
where $C_1^{A}$ and $C_1^{B}$ are values which can be computed from the previous equality.
\smallskip

\textbf{(b)}
We are given a general solution for the equilbrium interest rate as follows:
\eQb
r^* &=& \dfrac{\frac{1 + \rho_A}{2 + \rho_A}{Y_2}^A 
+ \frac{1 + \rho_B}{2 + \rho_B}{Y_2}^B}{
\frac{1}{2 + \rho_A}{Y_1}^A
+ \frac{1}{2 + \rho_B}{Y_1}^B} - 1.
\eQe
As we have $\rho_A = \rho_B = \rho$ assumption, we can further simplify the above solution as
\eQb
r^* &=& 
(1 + \rho) \dfrac{({Y_2}^A + {Y_2}^B)} {
({Y_1}^A + {Y_1}^B)} - 1.
\eQe

\smallskip

\textbf{(c)} 
Yes, $r^*$ is a function of total/aggregate endowments, as we can see from the above equation.
The real interest rate would not change if the endowments were redistributed.
Because the total is still the same.

\pagebreak

\textbf{(d)}
We first compute the saving of Anne. First note that we can do further derivation on $S_A$ 
and obtain
\eQb
S^* = \dfrac{1}{2 + \rho} ( Y_1 - \dfrac{1 + \rho}{1 + r^* } Y_2).
\eQe
Substituting the $r^*$ into the above equation, we have
\eQb
S_A^* = \dfrac{1}{2 + \rho} ( Y_1^A - \dfrac{1 + \rho}{
(1 + \rho) \frac{({Y_2}^A + {Y_2}^B)} {
({Y_1}^A + {Y_1}^B)}} Y_2^A).
\eQe
Simplifying further gives,
\eQb
S_A^* &=& \dfrac{1}{2 + \rho} ( Y_1^A - \dfrac
{({Y_1}^A + {Y_1}^B)} 
{({Y_2}^A + {Y_2}^B)} 
Y_2^A).
\eQe

The information given is not complete to deduce whether or not Anne will be saving or borrowing
in the new equilibrium. We need to know the relation between $Y_1$ and $Y_2$, to see that
second term in the saving computation is either positive or negative. The information is not given.
If we take $Y_1^A = Y$ as an implicit assumption, Anne will be borrwing as $Y_1^B$ will be higher,
causing the second term in the saving computation to be negative.

\smallskip

\textbf{(e)}
This is the same situation as the previous part. We can use the relation that $S = Y - C$ and
compute the $C$ as $C = Y - S$, using the above equation. We realize that this is a negative linear
relation with a $Y$ being a fixed y-intercept. Hence, Anne's consumption will increase with Bob's 
endowment increasing (saving will decrease with increase in $Y_B^1$ which will mean that the consumption
increases with a fixed endowment for Ann.

\end{solution}

\pagebreak

\begin{question}[2]
\end{question}
\begin{solution}
\textbf{(a)}
We are given a general solution for the equilbrium interest rate as follows:
\eQb
r^* &=& \dfrac{\frac{1 + \rho_A}{2 + \rho_A}{Y_2}^A 
+ \frac{1 + \rho_B}{2 + \rho_B}{Y_2}^B}{
\frac{1}{2 + \rho_A}{Y_1}^A
+ \frac{1}{2 + \rho_B}{Y_1}^B} - 1.
\eQe
Substituting $Y$ for all instances of endowment variables and simplifying, we obtain
\eQb
r^* &=& \dfrac{\frac{1 + \rho_A}{2 + \rho_A}
+ \frac{1 + \rho_B}{2 + \rho_B}}{
\frac{1}{2 + \rho_A}
+ \frac{1}{2 + \rho_B}} - 1.
\eQe
Multiplying both numerator and denominator yields
\eQb
r^* &=& \dfrac{(1 + \rho_A)(2 + \rho_B) + (1+\rho_B)(2 + \rho_A)}{
2 + \rho_B + 2 + \rho_A} - 1
\eQe
Expanding the denominator and simplying the numerator, we get
\eQb
r^* &=& \dfrac{4 + 3\rho_A + 3\rho_B + 2\rho_A \rho_B}
{4 + \rho_B + \rho_A} - 1.
\eQe
Combining the $-1$ term into the fraction, and factoring out $2$ we get
\eQb
r^* &=& 2\dfrac{\rho_A + \rho_B + \rho_A \rho_B}
{4 + \rho_B + \rho_A},
\eQe
as desired.

\bigskip

\textbf{(b)} Recall that the consumer actually saves if 
$\dfrac{Y_1}{Y_2} > \dfrac{1 + \rho}{1 + r}$. Therefore, 
as $\rho_A > r^*$ and $Y_1 = Y_2 = Y$, we have $1 < \dfrac{1+ \rho}{1 + r^*}$.
Hence, with the equilibrum interest rate Anne is a borrower in the first period. 

\end{solution}

\bigskip

\begin{question}[3]
\end{question}
\begin{solution}
\textbf{(a)}
We are given a general solution for the equilbrium interest rate as follows:
\eQb
r^* &=& \dfrac{\frac{1 + \rho_A}{2 + \rho_A}{Y_2}^A 
+ \frac{1 + \rho_B}{2 + \rho_B}{Y_2}^B}{
\frac{1}{2 + \rho_A}{Y_1}^A
+ \frac{1}{2 + \rho_B}{Y_1}^B} - 1.
\eQe
Substituting the given conditions into the equation gives,
\eQb
r^* &=& \dfrac{\frac{1 + \rho}{2 + \rho}Y}{
\frac{1}{2 + \rho}Y} - 1.
\eQe
Further simplification yields
\eQb
r^* &=& \rho.
\eQe
\textbf{(b)} As $r^* = \rho$, by substituting to the derived demand function,
we can see that Anne's consumption path is constant over time (i.e. ${C_1}^{A} = {C_2}^{B}$).
Bob's consumption path is also constant over time, as they share the same impatience.
\end{solution}


\end{document}











