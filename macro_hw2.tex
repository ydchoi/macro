\documentclass{article} % For LaTeX2e
\usepackage{nips14submit_e,times}
\usepackage{amsmath}
\usepackage{amsthm}
\usepackage{amssymb}
\usepackage{mathtools}
\usepackage{hyperref}
\usepackage{url}
\usepackage{algorithm}
\usepackage[noend]{algpseudocode}
%\documentstyle[nips14submit_09,times,art10]{article} % For LaTeX 2.09

\usepackage{graphicx}
\usepackage{caption}
\usepackage{subcaption}

\def\eQb#1\eQe{\begin{eqnarray*}#1\end{eqnarray*}}

\providecommand{\e}[1]{\ensuremath{\times 10^{#1}}}
\providecommand{\pb}[0]{\pagebreak}


\newenvironment{claim}[1]{\par\noindent\underline{Claim:}\space#1}{}
\newtheoremstyle{quest}{\topsep}{\topsep}{}{}{\bfseries}{}{ }{\thmname{#1}\thmnote{ #3}.}
\theoremstyle{quest}
\newtheorem*{definition}{Definition}
\newtheorem*{theorem}{Theorem}
\newtheorem*{question}{Question}
\newtheorem*{exercise}{Exercise}
\newtheorem*{challengeproblem}{Challenge Problem}
\newtheorem*{solution}{Solution}
\usepackage{verbatimbox}
\usepackage{listings}
\title{Intro to Macroeconmics: Problem Set II}


\author{
Youngduck Choi, Noah Gentile, Yuliya Takh \\
New York University \\
}


% The \author macro works with any number of authors. There are two commands
% used to separate the names and addresses of multiple authors: \And and \AND.
%
% Using \And between authors leaves it to \LaTeX{} to determine where to break
% the lines. Using \AND forces a linebreak at that point. So, if \LaTeX{}
% puts 3 of 4 authors names on the first line, and the last on the second
% line, try using \AND instead of \And before the third author name.

\newcommand{\fix}{\marginpar{FIX}}
\newcommand{\new}{\marginpar{NEW}}

\nipsfinalcopy % Uncomment for camera-ready version

\begin{document}


\maketitle

\begin{abstract}
This document contains the solutions to the problem set II.
\end{abstract}

\section{Solutions to the problems}

\begin{question}[1. Elasticity]
\end{question}
\begin{solution}
\textbf{(1)}
We first note that elasticity in general is defined by 
\eQb
\epsilon_{q,p} &=& \dfrac{\frac{\bigtriangleup q}{q}}{\frac{\bigtriangleup p}{p}}.
\eQe
The initial quantity, denoted as ${q_B}_{1}$, can be explicitly computed by
substituting the given values into the demand equation:
\eQb
{q_B}_{1} &=& 220 - 5(100) + \dfrac{1}{2}(1000) + 120 - 2(70), \\
&=& 200. 
\eQe
The new quantity demanded, denoted as ${q_B}_{2}$, can also be computed:
\eQb
{q_B}_{2} &=& 220 - 5(102) + \dfrac{1}{2}(1000) + 120 - 2(70), \\
&=& 190.
\eQe
Substituting the values into the elasticity equation, we have
\eQb
\epsilon_{q,p} &=& \dfrac{\frac{190-200}{200}}{\frac{102 - 100}{100}}, \\
&=& -\dfrac{5}{2}.
\eQe
Hence, the price-elasticity is $-2.5$.

\smallskip

\textbf{(2)}
The new quantity demanded can be computed as follows:
\eQb
{q_B}_{2} &=& 220 - 5(100) + \dfrac{1}{2}(1000) + 140 - 140 \\
&=& 220 
\eQe
Substituting the values into the elasticity equation, we have
\eQb
\epsilon_{q,p} &=& \dfrac{\frac{220-200}{200}}{\frac{140 - 120}{120}}, \\
&=& \dfrac{3}{5}.
\eQe
Hence, the cross-elasticity is $0.6$. As the cross-elasticity is positive,
Barro's and Mankiw's books are complements.

\pagebreak

\textbf{(3)}
The new quantity demanded can be computed as follows:
\eQb
{q_B}_{2} &=& 220 - 5(100) + \dfrac{1}{2}(1000) + 120 - 120 \\
&=& 220 
\eQe
Substituting the values into the elasticity equation, we have
\eQb
\epsilon_{q,p} &=& \dfrac{\frac{220-200}{200}}{\frac{60 - 70}{70}}, \\
&=& -\dfrac{3}{5}.
\eQe
Hence, the cross-elasticity is $-0.6$. As the cross-elasticity is negative, Barro's 
and the microeconomics books are complements.

\smallskip

\textbf{(4)}
The new quantity demanded can be computed as follows:
\eQb
{q_B}_{2} &=& 220 - 5(100) + \dfrac{1}{2}(1500) + 120 - 140 \\
&=& 470
\eQe
Substituting the values into the elasticity equation, we have
\eQb
\epsilon_{q,p} &=& \dfrac{\frac{470-200}{200}}{\frac{1500 - 1000}{1000}}, \\
&=& 2.7.
\eQe
Hence, the income elasticity is $2.7$. When income increases, the quantity 
demanded goes up, so it is a normal good.

\end{solution}

\bigskip

\begin{question}[2. Production]
\end{question}
\begin{solution}
\textbf{(1)}
We know that the Marginal Product of $L$ and of $K$ are respectively defined by
$\dfrac{\partial F}{\partial L } \> \text{ and } \> \dfrac{\partial F}{\partial K}.$ \\
Computing the above partials with the given production function, we obtain
\eQb
\dfrac{\partial F}{\partial L} &=& A, \\
\dfrac{\partial F}{\partial K} &=& B,
\eQe
where $A$ and $B$ are strictly positive constants.
Hence, the Marginal Product of $L$ is $A$ and the Marginal Product of $K$ is $B$.

\smallskip

\textbf{(2)}
With the above computation, we have shown that the marginal products are positive constants.
Hence, the marginal products are constants. It is decreasing, but not strictly decreasing.

\smallskip

\textbf{(3)}
The below figure contains the isoquants of production level 1 and 2 when $A = 2$ and $B = 5$.
\begin{figure}[h!]
  \caption{Isoquants of Production levels}
    \centering
  \includegraphics[width=0.5\textwidth]{iso.jpg}
\end{figure}

\end{solution}

\bigskip

\begin{question}[3. The Impact of a Minimum Wage law]
\end{question}
\begin{solution}
\textbf{(1)}

\smallskip

\textbf{(2)}

\smallskip

\textbf{(3)}


\end{solution}


\end{document}
