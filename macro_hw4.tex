\documentclass{article} % For LaTeX2e
\usepackage{nips14submit_e,times}
\usepackage{amsmath}
\usepackage{amsthm}
\usepackage{amssymb}
\usepackage{mathtools}
\usepackage{hyperref}
\usepackage{url}
\usepackage{algorithm}
\usepackage[noend]{algpseudocode}
%\documentstyle[nips14submit_09,times,art10]{article} % For LaTeX 2.09

\usepackage{graphicx}
\usepackage{caption}
\usepackage{subcaption}

\def\eQb#1\eQe{\begin{eqnarray*}#1\end{eqnarray*}}

\providecommand{\e}[1]{\ensuremath{\times 10^{#1}}}
\providecommand{\pb}[0]{\pagebreak}


\newenvironment{claim}[1]{\par\noindent\underline{Claim:}\space#1}{}
\newtheoremstyle{quest}{\topsep}{\topsep}{}{}{\bfseries}{}{ }{\thmname{#1}\thmnote{ #3}.}
\theoremstyle{quest}
\newtheorem*{definition}{Definition}
\newtheorem*{theorem}{Theorem}
\newtheorem*{question}{Question}
\newtheorem*{exercise}{Exercise}
\newtheorem*{challengeproblem}{Challenge Problem}
\newtheorem*{solution}{Solution}
\usepackage{verbatimbox}
\usepackage{listings}
\title{Intro to Macroeconmics: Problem Set IV}


\author{
Group 7: Youngduck Choi, Noah Gentile, Yuliya Takh \\
New York University \\
}


% The \author macro works with any number of authors. There are two commands
% used to separate the names and addresses of multiple authors: \And and \AND.
%
% Using \And between authors leaves it to \LaTeX{} to determine where to break
% the lines. Using \AND forces a linebreak at that point. So, if \LaTeX{}
% puts 3 of 4 authors names on the first line, and the last on the second
% line, try using \AND instead of \And before the third author name.

\newcommand{\fix}{\marginpar{FIX}}
\newcommand{\new}{\marginpar{NEW}}

\nipsfinalcopy % Uncomment for camera-ready version

\begin{document}


\maketitle

\begin{abstract}
This document contains the solutions to the problem set III.
\end{abstract}

\section{Solutions to the problems}

\begin{question}[1. RC Economy]
\end{question}
\begin{solution}
\textbf{(1)} We are given the RC economy such that RC preferences over $C$ and $L$ is
given by 
\eQb
U(C,L) = -\dfrac{1}{3}C^{-3} - \dfrac{1}{2}L^2,
\eQe
which can be re-written in terms of $C$ and $O$ as
\eQb
U(C,O) = -\dfrac{1}{3}C^{-3} - \dfrac{1}{2}(1-O)^2,
\eQe
and the production technology is given by 
\eQb
F(K,L) &=& AK^{\frac{1}{3}}L^{\frac{2}{3}}.
\eQe
With $K$ fixed, we want to find the solution to the RC problem.
We first compute the tangency condition of $MRS_{\text{O for C}} = MPL$. To compute $MRS$ on the
LHS, we first take the partials with respect to first and second variable, obtaining
\eQb
U_{1}(C,O) &=& C^{-4} \\
U_{2}(C,O) &=& 1-O \\
&=& L.
\eQe
Hence, we have that
\eQb
MRS_{\text{O for C}} &=& \dfrac{U_{2}(C, O)}{U_{1}(C,O)} \\
&=& L{C^4}.
\eQe

Computing the MPL, we obtain
\eQb
MPL &=& F_{L}(K,L) \\
&=& \dfrac{2}{3}AK^{\frac{1}{3}}L^{-\frac{1}{3}}.
\eQe
Therefore, via substitution, we have the tangency condition as
\eQb
LC^4 &=& \dfrac{2}{3}AK^{\frac{1}{3}}L^{-\frac{1}{3}}.
\eQe
With the constraint is given by $C = F(K,L)$, by substitution we have the explicit constraint as
\eQb
C &=& AK^{\frac{1}{3}}L^{\frac{2}{3}}.
\eQe
Substituting the above equation into the tangency condition and re-arranging, we have 
\eQb
L = \dfrac{2}{3}\dfrac{AK^{\frac{1}{3}}L^{-\frac{1}{3}}}{A^4K^{\frac{4}{3}}L^{\frac{8}{3}}}. 
\eQe
Simplifying gives
\eQb
L &=& \dfrac{2}{3}A^{-3}K^{-1}L^{-3}. \\
\eQe
Moving $L$ on the RHS to LHS and explictly solving for it, we get
\eQb
L^* &=& (\dfrac{2}{3}A^{-3}K^{-1})^{\frac{1}{4}}.
\eQe
So, the consumption is,
\eQb
C^* &=& AK^{\frac{1}{3}}{(\dfrac{2}{3}A^{-3}K^{-1})}^{\frac{1}{4}} \\
&=& (\dfrac{2}{3})^{\frac{1}{4}}A^{\frac{1}{4}}K^{\frac{1}{12}} \\
&=& (\dfrac{2}{3}AK^{\frac{1}{3}})^{\frac{1}{4}}.
\eQe
Hence, we have computed the solution to RC's problem.

\smallskip

\textbf{(2)}
$L$ is not increasing in $A$. It is, in fact, decreasing as we have $A^{-3}$ in the computed solution.
L is an inferior good. Hence, the income effect is $C$ goes up and $O$ goes down, therefore $L$ goes up.
The opportunity cost of leisure became relatively more expansive, so RC substitutes toward $C$.
\smallskip

\textbf{(3)}
The effect of a natural disaster, which translates to reduction of the quantity $K$, will decrease
the consumption, but will increase the labor. This comes from a direct relationship 
we have from the computed solutions above.
\end{solution}
\end{document}

